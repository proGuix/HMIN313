approche par les graphes

Les données

Dictionnaire
Graphe de données (liste des prédicats triée)
Arbre de préfixe des relations selon les prédicats
Statistiques simples (optimisation): chaque prédicat connait son nombre d'occurence dans les données ce qui diminue l'espace de recherche des sujets initiaux.
Arbre de voisinage (FP-TREE) (arbre trié en largeur et en profondeur sur les prédicats) pour chaque sujet des données, optimisation avec la notation begin/end (savoir si n1 est voisin avec n2 est en complexité linéaire sur le nombre de prédicats de n1).

La requete

La requete est composé de plusieurs branche, on peut la voir comme un graphe étoilé.
On fusionne les branches si deux ont le meme objet.
on trie la liste des prédicats de chaque branche qui ont plus d'un prédicat, ce trie est important car l'algorithme de résolution va faire correspondre la liste de prédicats d'une des branches de la requete avec l'arbre des préfixes des relations.
Il faut donc d'abord trier nos branches pour savoir laquelle nous allons commencer à faire correspondre avec les données. Pour cela nous utilisons une heuristique qui va trier nos branches sur la sélectivité des prédicats. Nous savons le nombre d'occurrence de chaque prédicats dans les données. Notre heuristique récupère d'abord les prédicats distincts de chaque branche et les trie de la plus petite à la plus grande occurence. L'heuritique choisit en premier les branches qui contiennent le prédicat le moins occurrent et les trie par longueur et on répète cette étape sur la liste des prédicats distincts et trié. Cela nous donne un ordre non total sur la sélectivité des prédicats mais un ordre de trie correct en un temps quadratique sur le nombre de branches au lieu d'un ordre parfait en un temps cubique sur le nombre de prédicats distincts.
Une fois que les branches de la requetes sont triés, on prend la première, on trouve tous les sujets qui possède une relation contenant les prédicats de la première branche grace à l'arbre des préfixes des relations. Pour tous ces sujets, on vérifie qu'il contient la branche courante (correspondance avec les prédicats et l'objet de la branche de la requete étoile) grace à l'abre de voisinage (FP-TREE). Si c'est le cas alors on vérifie que ce sujet contient aussi toutes les autres branches de la requete. Si c'est le cas, le sujet est un réultat de la requete étoile.

Nous avons passer du temps à implémenter une solution naive pour le FP-TREE du voisinage mais suite à une optimisation de notre professeur avec le systeme de notation begin/end qui simplifiait la recherche des voisins à des comparaisons numériques. Notre solution initial était de parcourir les branches du FP-TREE, nous avons donc totalement recodé cette partie. Une fois que nous avions les linkedList (les fléches rouges sur le cours). La recherche des voisins selon une liste de prédicats se faisait en une complexité égale à la somme des tailles des lindekList des prédicats concernés. Nous avons pensez à une optimisation en rajoutant à l'algorithme une recherche dichotomique selon la distribution des données, malheureusement cela ne s'est pas fait par manque de temps et une étude supplémentaire aurait encore améliorer la rapidité de l'algorithme.
